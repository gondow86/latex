\documentclass[dvipdfmx]{article}
\usepackage[dvipdfmx]{graphicx}
\usepackage{amsmath, amssymb}
\usepackage{mathtools}
\usepackage{here}
\begin{document}
\title{Weekly Report}
\author{Riku Gondow}
\maketitle
\section{Progress}
\begin{itemize}
    \item Apply wavelet reconstruction after applying EMD 
    \item Check accuracy with different IMFs to be selected in noisy dataset under close-set condition
\end{itemize}
\section{Results}

\begin{table}[H]
    \caption{Accuracy when using EMD + Wavelet reconstruction}
    \centering
    \begin{tabular}{c|c}
    \hline
    IMF4, 5, 6 & 32.48\% \\
    IMF5, 6 & 34.26\% \\
    IMF6 & 34.31\% \\
    \hline
    \end{tabular}
\end{table}

The accuracy improves by 2.7 ~ 2.8\% when not using IMF4. I think this is because IMF4 contains more high-frequency components and noise than the heart rate signal compared to IMF5 and 6.

\begin{table}[H]
    \caption{Comparison of EMD vs EMD + Wavelet}
    \centering
    \begin{tabular}{c|c}
    \hline
    Only EMD & 29.83\% \\
    EMD + Wavelet rec & 32.48\% \\
    \hline
    \end{tabular}
\end{table}


Since the accuracy when only BPF is applied is 47.27\%, it is possible that EMD has a bad influence. 

Also, compared to the case where only EMD is applied, the wavelet transform increases the accuracy by 2.6\%, suggesting that the wavelet transform may contribute to the improvement of accuracy to some extent.

\section{Next Plan}
\begin{itemize}
    \item Try Another preprocessing method
    \begin{itemize}
        \item BPF + Wavelet reconstruction
        \item Using I/Q data and applying ellipse fitting (now just using I data)
        \item Adding threshold processing to wavelet coefficients
        \item Changing the decomposition level of wavelet transform (It affects the rate of denoising.)
    \end{itemize}
\end{itemize}
\end{document}